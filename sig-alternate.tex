%% Thesis template
%% PMC, University of Minho
%% if required, comment unused packages
\documentclass[12pt,english]{book}
% \usepackage[portuguese]{babel}
% \usepackage[utf8]{inputenc} 
\usepackage{times}
\usepackage[T1]{fontenc}
\usepackage[latin1]{inputenc}
\usepackage{a4wide}
\usepackage{fancyhdr}
\pagestyle{fancy}
\usepackage{subfigure}
\usepackage{float}
\usepackage{graphicx}
\usepackage{setspace}
\usepackage{cite}\onehalfspacing

\makeatletter

%%%%%%%%%%%%%%%%%%%%%%%%%%%%%% LyX specific LaTeX commands.
%% Bold symbol macro for standard LaTeX users
\newcommand{\boldsymbol}[1]{\mbox{\boldmath $#1$}}

\floatstyle{ruled}
\newfloat{algorithm}{tbp}{loa}
\floatname{algorithm}{Algorithm}

%%%%%%%%%%%%%%%%%%%%%%%%%%%%%% Textclass specific LaTeX commands.
 \usepackage{verbatim}
 \newenvironment{lyxlist}[1]
   {\begin{list}{}
     {\settowidth{\labelwidth}{#1}
      \setlength{\leftmargin}{\labelwidth}
      \addtolength{\leftmargin}{\labelsep}
      \renewcommand{\makelabel}[1]{##1\hfil}}}
   {\end{list}}

%%%%%%%%%%%%%%%%%%%%%%%%%%%%%% User specified LaTeX commands.
\usepackage{color}
\usepackage{colortbl}
\usepackage{url}
\input{epsf}
\newcommand{\single}{\renewcommand\baselinestretch{1.0}}
\newcommand{\onehalf}{\renewcommand\baselinestretch{1.5}}
\newcommand{\double}{\renewcommand\baselinestretch{2.0}}
\fancyhead{}
\fancyhead[LE]{\slshape \leftmark}
\fancyhead[RO]{\slshape \rightmark}
\cfoot{\thepage}
\setlength{\parskip}{2mm}

\usepackage{babel}
\makeatother
\begin{document}
\begin{minipage}[c]{1.0\columnwidth}%
\begin{doublespace}
\vspace{2cm}
\begin{center}{\huge A new Framework to enable rapid innovation in Cloud Datacenter through a SDN approach.}\end{center}{\huge \par}
\end{doublespace}

\vspace{2cm}
\begin{center}{\large Jos\'{e} Teixeira}\end{center}{\large \par}
\vspace{2cm}

\begin{quote}
\begin{center}A thesis submitted to the University of Minho in the
subject of Informatics, for the degree of
% or Doctor of Philosophy 
Master of Science, under scientific supervision of Prof. Stefano Giordano and Prof. Alexandre Santos\end{center}\vspace{3cm}

\end{quote}
\begin{singlespace}
\begin{center}University of Minho\end{center}

\begin{center}School of Engineering\end{center}

\begin{center}Department of Informatics\end{center}
\end{singlespace}

\begin{center}{\large September, 2013}\end{center}\end{minipage}%
\thispagestyle{empty}

\pagenumbering{roman}


\newpage


\chapter*{Acknowledgments}

\addcontentsline{toc}{chapter}{Acknowledgments}

\noindent I would like...

\medskip{}
\noindent I also...

\chapter*{Abstract}

\addcontentsline{toc}{chapter}{Abstract}

\begin{singlespace}
In the last years, the widespread of Cloud computing as the main paradigm to deliver a large plethora of virtualized services significantly increased the complexity of Datacenters management and raised new performance issues for the intra-Datacenter network.
Providing heterogeneous services and satisfying user\'s experience is really challenging for Cloud service providers, since system (IT resources) and network administration functions are definitely separated.

In this scenario, a recent approach to programmable networks (i.e., Software-Defined Networking - SDN) seems to be a promising way to satisfy DC network requirements\cite{ibmnec}.
SDN based architecture decouples control and data planes: the most deployed SDN protocol is OpenFlow (OF)\cite{onf}\cite{openflow}, which allows to set into OF compliant switches forwarding rules established by a centralized intelligence called controller. 

Since SDN allows to re-define and re-configure network functionalities (possibly up to the physical layer), the basic idea is to introduce a new framework that allows to develop and test new policies that enables a more efficient, agile, scalable and simple use of both server and network resources.

More specifically, the framework enhances both POX (an OF controller written in python) and Mininet (a well-known SDN emulator), with all the extensions necessary to experiment novel control and management strategies of IT and network resources.

... talk about obtained results and conclusions(not finished yet)

\end{singlespace}

\paragraph{Keywords:}
Datacenter, Cloud, SDN, OpenFlow.

\newpage

\cleardoublepage

\addcontentsline{toc}{chapter}{Contents}\tableofcontents{}

\cleardoublepage

\chapter*{List of Acronyms}

\addcontentsline{toc}{chapter}{List of Acronyms}

\markboth{LIST OF ACRONYMS}{LIST OF ACRONYMS}

\begin{lyxlist}{00.00.0000}
\begin{singlespace}
\item [DC]Datacenter
\item [SDN]Software Defined Networking
\item [OF]Openflow
\item [VM]Virtual Machine
\item [IP]Internet Protocol 
\item Add as needed...
\end{singlespace}
\end{lyxlist}

\cleardoublepage

\addcontentsline{toc}{chapter}{List of Figures}\listoffigures

\cleardoublepage

\addcontentsline{toc}{chapter}{List of Tables}\listoftables

\cleardoublepage

\setcounter{page}{0}

\pagenumbering{arabic}


\chapter{Introduction\label{cha:introduction}}

\section{Introduction}

A Cloud Datacenter consists of virtualized resources that are dynamically allocated, in a seamless and automatic way, to a plethora of heterogeneous applications.
In Cloud Datacenters, services are no more tightly bounded to physical servers, as occurred in traditional DCs, but are provided by Virtual Machines that can migrate from a physical server to another, increasing both scalability and reliability.

Software virtualization technologies allow a better usage of DC resources; DC management, however, becomes much more difficult, due to the strict separation between systems (\textit{i.e.}, servers, VMs and virtual switches) and network (\textit{i.e.}, physical switches) administration.
Moreover, new issues arise, such as isolation and connectivity of VMs.
Services performance may suffer from the fragmentation of resources as well as the rigidity and the constraints imposed by the intra-DC network architecture (usually a multilayer 2-tier or 3-tier fat-tree composed of Edge, Aggregation and Core switches\cite{dc_arch}).
Therefore, Cloud service providers (\textit{e.g.},\cite{amazon}) ask for a next generation of intra-DC networks meeting the following features: 1) efficiency, \textit{i.e.}, high server utilization; 2) agility, \textit{i.e.}, fast network response to server/VMs provisioning; 3) scalability, \textit{i.e.}, consolidation and migration of VMs based on applications' requirements; 4) simplicity, \textit{i.e.}, performing all those tasks easily\cite{baldonado}.

In this scenario, a recent approach to programmable networks (\textit{i.e.}, Software-Defined Networking) seems to be a promising way to satisfy DC network requirements\cite{ibmnec}. 

NOTE: To agressive this next sentense paragraph

SDN--based architecture decouples control and data planes: the most deployed SDN protocol is OpenFlow (OF)\cite{openflow}\cite{onf}, which allows to set into OF--compliant switches forwarding rules established by a centralized intelligence called controller.
Since SDN allows to re-define and re-configure network functionalities (possibly up to the physical layer), the basic idea is to introduce an SDN cloud-DC controller that enables a more efficient, agile, scalable and simple use of both VMs and network resources. Nevertheless, before deploying the novel architectural solutions, huge test campaigns must be performed in experimental environments reproducing a real DC.

To this aim, we introduce a novel framework that enhances both Mininet\cite{mininet} and POX\cite{pox} with all the software modules necessary to emulate an SDN-based intra-DC network, such as DC topology discovery, network traffic generation, etc. Specifically designed for DC environments, our framework allows to develop and assess novel SDN-Cloud-DC controllers, and to compare the performance of control and management strategies jointly considering both IT and network resources\cite{im2013}. It is worth highlighting that the developed software modules may be ported in a real controller without changes, as our framework inherits such basic feature from Mininet.

The rest of the paper is organized as follows: section \ref{sec:rw} provides a short survey of related works, whereas section \ref{sec:dc} details the architecture and the functionalities of our framework. Section \ref{sec:usec} presents an use case while section \ref{sec:perf} evaluates the performance of the framework. Finally, section \ref{sec:conc} concludes the paper with some final remarks.

\begin{itemize}
	\item DataCenter
	\item Cloud
	\item Cloud DataCenter
	\item SDN - Software define Networks
	\item Openflow
	\item ...
\end{itemize}

% \pagebreak
\section{Motivation and objectives\label{sec:motobj}}

\begin{itemize}
	\item Understanding the basic features of SDN paradigm
	\item Studying the problematics in cloud DC VM allocations
	\item Apply the SDN paradigm to better exploit the DC resources
	\item Develop a framework for Cloud Datacenter emulation and new VM allocation policies
	\item ...
\end{itemize}

\section{Dissertation layout}

In the present Chapter \ref{cha:introduction} - ...

\chapter{State of art \label{cha:stateofart} }

Usually background and related work ...

\section{Available solutions}

Write something generic

\subsection{CloudSim}

\subsection{NetFPGA Emulation}

\subsection{Meridian}

\subsection{Networkcloudsim}

\subsection{Greencloud}

\subsection{icancloud}

\newpage
\section{Virtual Machine Allocation Policies}

\newpage

% Uncomment to include file.pdf
%\begin{figure}%[H]
%\begin{center}\includegraphics[scale=0.8]{file}\end{center}
%\caption{Legend \label{fig:LABEL}}
%\end{figure}


\chapter{Architecture and design \label{cha:arqdes} }

Conceptual view and architecture of the proposed solution (implementation details can go into a different chapter, if required)...

\section{Framework architecture}

Generically talk about the architecture...


\section{Framework modules: Mininet}

\subsection{Topology Generator}

\subsection{Traffic Generator}

Describe each module, it's functionalities, limitations, how it can be used/improved (improved if the user wants to add new features)

\begin{itemize}
	\item Talk generally about the traffic generator
	\item Talk about the one's we tried (pros and cons)
\end{itemize}

\newpage


\section{Framework modules: Controller}

Describe each module, it's functionalities, limitations, how it can be used/improved (improved if the user wants to add new features)

\subsection{Topology Discovery}

\subsection{OF Rules Handler}

\subsection{Statistics Handler}

\subsection{VM Request Handler}

\subsection{VMM - Virtual Machines Manager}

\subsection{Network Traffic Requester}

\subsection{POX Modules}

\subsection{User Defined Logic}

\newpage


\section{Framework modules: Web Platform}

Describe each module, it's functionalities, limitations, how it can be used/improved (improved if the user wants to add new features)

\newpage


\section{Framework modules: VM Requester}

Describe each module, it's functionalities, limitations, how it can be used/improved (improved if the user wants to add new features)

\newpage


\section{Using the framework}

\subsection{Emulator}

Describe how to use the framework (emulation part) and how to access the API..

\subsection{Real Environment}

Describe what changes in the real environment (the modules that are disabled and the ones that need to be enabled)

\chapter{Framework extensions \label{cha:} }

\section{Enabling QoS}
\subsection{State of art: QoS in SDN}
\subsection{QoS in the framework}
\newpage

\section{Enabling Virtual Machine migration}
\subsection{State of art: Virtual Machine Migration Policies}
\subsection{Virtual Machine migration in the framework}
\subsection{Usecase}

\chapter{Validation and tests \label{cha:valtes} }

Usually test and validation of the proposed solution ...

\section{Framework Validation}

\begin{itemize}
	\item Show how Bf goes against WF with server driven algorithm (show server occupation)
	\item Show how Bf goes against WF with network driven algorithm (show network occupation) (although the behaviour is similar is allow to say that net algorithm may use switch statistics)
\end{itemize}
\newpage


\section{Performance Evaluation}

Get the tests from the submitted paper.
\newpage

\section{Real environment tests}
\begin{itemize}
	\item Talk about the environment which was setup
	\begin{itemize}
		\item Chosen hypervisor
		\item Talk about Xen api and the alternative solution (ssh each server and run a script to clone the vm)
		\item OpenVswitches VS NetFPGA problems
		\item 
	\end{itemize}
\end{itemize}
\newpage

\chapter{Conclusions\label{cha:conclusions}}

This chapter provides ...

\section{Main contributions}

\section{Future work}

\appendix

\chapter{Name of the Appendix}

\cleardoublepage

\markright{\slshape Appendix}

\cleardoublepage
\bibliographystyle{unsrt}
\addcontentsline{toc}{chapter}{\bibname}

%% Add file.bib
\bibliography{sigproc}
\nocite{*}



\end{document}
